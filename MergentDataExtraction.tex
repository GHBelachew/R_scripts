# =====================================================================
# R Scripting
# This a template script to extract Mergent Intellect Data for Data requests.
#======================================================================

# Loading libraries == Note: some libraries will may not be needed but keep the list
library(data.table)
library(rgdal) # Bindings for the Geospatial Data Abstraction Library
library(raster) # Geographic Data Analysis and Modeling
library(rgeos) # Interface to Geometry Engine - Open Source (GEOS)
library(ggmap)
library(ggplot2)
library(sf)
library(sp)
library(maptools) # to create shape file
library(tmap)
library(dplyr)
library(reshape2)
library(reshape)
library(plyr)

#=========================================================================
# Reading Data ; path of data file and name of data source are subject to change

df <- read.csv("Path/SourceData.csv")

#========================================================================    
#Data Preprocessing and Prep
# Change employment from factor to character and then to numeric by removing the comma

df$EmThisS <- as.character(df$EmThisS)
df$EmThisS <- gsub(",", "", df$EmThisS)
df$EmThisS <- as.numeric(df$EmThisS)

# change SalesY1 from factor to character and then to numeric by removing the comma and Dollar sign

df$SalesYR1 <- as.character(df$SalesYR1)
df$SalesYR1 <- gsub(",", "", df$SalesYR1)
df$SalesYR1 <- gsub("\\$", "", df$SalesYR1)
df$SalesYR1 <- as.numeric(df$SalesYR1)
#========================================================================
# Extract needed fields for the request

df2 <- df[ , c("ID",
                          "DUNNoV",
                          "Company",
                          "LineOfBu",
                          "StAdrs",
                          "StCity",
                          "StState",
                          "StZip",
                          "MailAdrs",
                          "MailCity",
                          "MailSta",
                          "MailCo",
                          "MailZip",
                          "PrmNAICS",
                          "PrNAICSD",
                          "EmThisS",
                          "SalesYR1",
                          "FinalLat",
                          "FinalLon",
                          "COName")]
# Extract needed counties (This list is dependent on the request
df3 <- subset(df2,df2$COName %in% c("County1", "County2","County3","County4"))

## Testing Summaries (count of number of business and employment)
# setDT(df3)
# df3[, .(`Count` = .N), by = COName]
# aggregate(EmThisS ~ COName, data=df3, FUN=sum)

#==========================================================================
# Mapping the file,  to check if the coordinates are correct
# get zone shape file(merlam shape file)
Zone <- readOGR(dsn = "Path", layer = "shafilename") 
# set projection to WGS84
Zoneproj <- spTransform(zone,CRS("+proj=longlat +ellips=WGS84"))


df3_plt <- SpatialPointsDataFrame(df3[ ,c("FinalLon","FinalLat")], df3,
                                   proj4string = CRS("+proj=longlat +ellips=WGS84"))

plot(zoneproj)
plot(df3_plt,add=TRUE)
							   
#==========================================================================
# Export data and shape
write.csv(df3, "Path/OutPutName.csv", row.names = F)
writeOGR(df3_plt, layer = 'OutPutName','Path',driver="ESRI Shapefile")

# Add other commands as needed
#===================================END====================================
