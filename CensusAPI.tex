## Open the package we need
library(XML) ## Tools for Parsing and Generating XML Within R and S-Plus
library(RCurl) ## General Network (HTTP/FTP/...) Client Interface for R
library(httr) ## Tools for Working with URLs and HTTP
library(stringr) ## Simple, Consistent Wrappers for Common String Operations
library(rjson) ## JSON for R
library(tm) ## Text Mining Package
library(RJSONIO) ## Serialize R objects to JSON, JavaScript Object Notation

## What datasets we can download:
CensusAPI_url <- 'http://api.census.gov/data.html'
CensusAPI_urldata <- GET(CensusAPI_url)
datasets <- readHTMLTable(rawToChar(CensusAPI_urldata$content), stringsAsFactors = FALSE)
datasets <- as.data.frame(datasets)
colnames(datasets) <- c("Title", "Description", "Vintage", "Dataset Name", "Geography List", 
                        "Variable List", "Tag List", "Examples", "Developer Documentation", "API BASE URL")
write.csv(datasets, "Census API_Datasets and its descendants.csv", row.names = FALSE, na = "")
print("Please run the code and check the Console, and remember the index.")
print(datasets$Title)

## Tell me the index
##### !!!!! What you need to type:
index <- 23
basicurl <- datasets$`API BASE URL`[index]

variableurl <- paste(basicurl, '/variables.html')
variableurl <- str_replace_all(variableurl, " ", "")## remove the space in variableurl
variableurl_data <- GET(variableurl)
variables <- readHTMLTable(rawToChar(variableurl_data$content), stringAsFactors = FALSE)
variables <- as.data.frame(variables)
colnames(variables) <- c("Name", "Label", "Concept", "Required", "Predicate")
variables$Name <- as.character(variables$Name)
variables$Label <- as.character(variables$Label)
variables$Concept <- as.character(variables$Concept)
variables$Required <- as.character(variables$Required)
variables$Predicate <- as.character(variables$Predicate)
print("Please open the table and remember the index.")## check the table 

## Tell me the index
##### !!!!! What you need to type
variable_index <- c(7,8,9) 
variablesubset <- variables[rownames(variables) %in% variable_index,]
variables <- variablesubset$Name
variables_name <- ""
## create the variable
for (i in 1:length(variables)) {
  variables_name <- paste(variables_name,",",variables[i])
}
variables_name <- str_replace_all(variables_name, " ", "")
variables_name <- substr(variables_name,2, nchar(variables_name))

## Tell me the geography SPC ## default
geography <- 'county:003,007,019,051,059,063,073,125,129&in=state:42'

###key
##### !!!!! What you might need to change
key <- 'f9f0f32785cb14f24152fa86c38fd7cf239e8a9c'
## generate the url
url <- paste(basicurl,'?get=',variables_name,'&for=',geography,'&key=',key)
## delete the space in url
url <- str_replace_all(url, " ", "")

## from json to csv
json_file <- fromJSON(url)
## us NA instead null
json_file <- lapply(json_file, function(x) {
  x[sapply(x, is.null)] <- NA
  unlist(x)
})
data <- do.call("rbind", json_file)
colnames(data) <- c(data[1,])
## remove the first row
data <- data[-1,]
## change the colname from Name to Label
for (i in 1:length(variable_index)) {
  colnames(data)[i] <- c(variablesubset$Label[i])
}

## export as csv
vname <- ""
for (i in 1:length(variables)) {
  vname <- paste(vname,"",variables[i])
}
filename <- paste(datasets[index,1],vname,"SPC")
## remove puctuation
filename <- removePunctuation(filename, preserve_intra_word_dashes = FALSE)
filename <- paste(filename,'.csv')
write.csv(data, filename, row.names = FALSE, na="")
