# create shape point shape file usig xy coordinates
library(data.table)
library(rgdal) # Bindings for the Geospatial Data Abstraction Library
library(raster) # Geographic Data Analysis and Modeling
library(rgeos) # Interface to Geometry Engine - Open Source (GEOS)
library(ggmap)
library(data.table)
library(ggplot2)
library(sf)
library(sp)
library(maptools) # to create shape file
library(tmap)

#============= create shape file from final Lat and final Lon =======
 
# Import data 
df <- read.csv("L:/Path/file.csv", stringsAsFactors = FALSE)
head(df)
names(df)
str(df)
head(df)

# Reading Excel file
 
 
 
 

# subset data that is needed to create shape file ================

MI20_FXY <- AllPartsData_final[c("id",
                                             "Company.Name",
                                             "Company.Type",
                                             "Subsidiary.Status",
                                             "Owns.Rents",
                                             "Latitude",
                                             "Longtitude")




# because Employment had comma in the number it is imported change it as character.
# comma should be eliminate and change character to numeric
# using gsub command eliminate the comma from employment 

MI20_FXY$EmThisS <- gsub(",", "",MI20_FXY$EmThisS) # replace the comma in employment with nothing
str(MI20_FXY)
MI20_FXY$EmThisS <- as.numeric(MI20_FXY$EmThisS) # Change employment from character to numeric
MI20_FXY_Loc <- SpatialPointsDataFrame(MI20_FXY[ ,c(16,15)], MI20_FXY,
                                   proj4string = CRS("+proj=longlat +ellips=WGS84"))
names(MI20_FXY_Loc)
# write the point files to shape files

writeOGR(MI20_FXY_Loc, layer = 'MI20_FXY_Loc','L:/Mergent Intellect employment database/2020 Jan/GIS/Shape',driver="ESRI Shapefile")


#==================================== Not used Start======================================================
# Reading Merlam zone shape file 
MerlamZ <- readOGR(dsn = "L:/Mergent Intellect employment database/2018 Jan/Compilation/R/Shape", layer = "TAZ_Cycle10_37county_Onlyin37County") # this shape file comes from the study area locaton 
plot(MerlamZ)

# redefine projection
MerlamZproj <- spTransform(MerlamZ,CRS("+proj=longlat +ellips=WGS84")) # set projection to WGS84
plot(MerlamZproj)
# intersecting Mergent point with 37County merlam zones

MI20_FXY_Loc_MZ <- intersect(MI20_FXY_Loc, MerlamZproj) 

tm_shape(MerlamZproj) + tm_borders(alpha = .4) +
  tm_shape(MI20_Loc) + tm_dots(col = "red", size = .05) +
  tm_layout(title = "Businesses in  37 County Ara", title.size = .75) +
  tm_layout(title = "Mergent  Intellect Business in 37 County Area ", title.size = .75)
# creata data frame
# select fields needed 
# export to CSV

MI20_37CountyData <- as.data.frame(MI20_FXY_Loc_MZ)
colnames(MI20_37CountyData)

MI20_37CountyDataFXY <- MI20_37CountyData[ ,c(1:16,20)]
head(MI20_37CountyDataFXY)
#Export 37County Business with Zone to csv
write.csv(MI20_37CountyDataFXY, "L:/Mergent Intellect employment database/2020 Jan/Compilation/R/Output/MI20_37CoData_MZ.csv", row.names = FALSE)

